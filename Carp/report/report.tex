\documentclass[conference,compsoc]{IEEEtran}
\usepackage{CJK}
\ifCLASSOPTIONcompsoc
  \usepackage[nocompress]{cite}
\else
  \usepackage{cite}
\fi
\ifCLASSINFOpdf
\else
\fi
\usepackage{amsmath}
\usepackage{algorithm}
\usepackage{algorithmicx}
\usepackage{algpseudocode}
\usepackage{booktabs}
\renewcommand{\algorithmicrequire}{\textbf{Input:}}
\renewcommand{\algorithmicensure}{\textbf{Output:}}
\usepackage{url}
\hyphenation{op-tical net-works semi-conduc-tor}
\begin{document}
\title{CARP Solver}
\author{\IEEEauthorblockN{Peijun Ruan  11610214}\\
\IEEEauthorblockA{School of Computer Science and Engineering\\
Southern University of Science and Technology\\
Email: 11610214@mail.sustc.edu.cn}}
\maketitle
\IEEEpeerreviewmaketitle
\section{Preliminaries}
\subsection{Introduction}
 The Capacitated Arc Routing Problem(CARP) problem is a classical combinatorial optimization problem.In this project, the problem can be described as: given a strong connected graph,each edge has a cost,and some of edges need to be serviced.Here are some number of vehicles to satisfy the demand which all start at a specific position(depot) and after finish its task, it will go back that position.what should be noticed is that each edge with demand should only be serviced once and each vehicle has a capacity to limit the demand it can satisfy.

 So in this problem, there are three constraints:
 \begin{itemize}
 \item each vehicle should start from depot and go back depot at last.
 \item each demand edge should be serviced and serviced only for once.
 \item the total demand of edges a vehicle serviced should not exceed its capacity.
 \end{itemize}

 CARP has been proved to be a NP-hard problem and the time need to solve is $\mathcal{O}(n!)$,  which means it is impossible to find the best solution in a polynomial time.Then the goal focus on: in given time, find a solution which satisfy all three constraints above and has minimum cost.
\subsection{Software}
This project is written in Python 3.6 as the guide document requires. I use Terminal to run and test this program and Pycharm to write and debug.The outer library I use in this project is Numpy and the built-in libraries I use include getopt, time, random, multiprocessing, sys and math.
\subsection{Algorithm}
In this project, I choose to use Dijkstra algorithm to build a two-dimension array for the graph and use Path-Scanning to build a original solution, then use Simulated Annealing[1] algorithm to change the original solution and formulate a new solution, comparing the new solution with the old one, if better then replace the old one with new one, else replace the old one with the new worse one with a certain possibility. After that ,repeat the solution formulation and replacement steps until the time is run out or the solution doesn't change for certain times, like 8 times, then return the final solution.
\section{Methodology}
In this section, I will show the method I use specifically and show how I represent the data in data structure.
\subsection{Representation}
\textbf{Graph representation}\\
\begin{itemize}
    \item edge\_cost: a two-dimension numpy array, store the cost from one vertex to another. If two vertices are not connected directly, the value will be 9999999.\\

    \item graph: a two-dimension numpy array, store the value of demand edge. If two vertex form a demand edge, the value should be larger than 0, if only form a edge, the value should be 0, if they are not connected directly, the value will be small than 0.\\

    \item min\_cost: a two-dimension numpy array, store the minimum cost from one vertex to any other.\\
\end{itemize}

\textbf{Answer representation}\\
\begin{itemize}
    \item solution: a list whose elements are lists, each vehicle's route are represent as a list consisted of tuples, and combine all the vehicles' list to a list. That's the solution of this project.
\end{itemize}.
\subsection{Functions}
Here shows the main functions I use in the carp\_solver.py
\begin{itemize}
    \item get\_input: get the input from Terminal and store the given data in variables.\\
    \item get\_graph: read the data in given file and build the graph.\\
    \item dijkstra: find the shortest cost from one vertex to any others.\\
    \item get\_min\_cost: use dijkstra to generalize the min\_cost array.\\
    \item path\_scanning: use some strategy to find a valid original solution.\\
    \item fitness: calculate the cost of given solution.\\
    \item operators: the methods to create a new solution base on the original solution.
        \begin{itemize}
            \item swap: exchange two elements.
            \item insertion: choose one element in the solution and randomly insert it into one position.
            \item MS[2]: the Merge-Split operator that use path\-scanning to create a new solution.
        \end{itemize}
    \item sub\_thread: simulated annealing\\
\end{itemize}

\subsection{Detail of Algorithm}
Here show the psudocode of the main functions. \\

\begin{algorithm}
    \caption{Dijkstra}
    \begin{algorithmic}[1]
        \Require $vertex\ v$ ,\ $graph\ G$
        \Ensure $one-dimension\ array$ $ min\_cost$
        \Function{Dijkstra}{$G\, ,v$} \\
        $min\_cost \gets [ \ ]$ , $min\_cost[v] \gets 0$
        \For{$vertex \ s \ $in \ $G $}
            \State $min\_cost \ \gets G[v][s] $
        \EndFor
        \For{$i$ in range $vertices' \ number$}
            \State $min \gets 9999999$
            \State $p \gets closest \ vertex \ to \ v \ that \ unvisited $
            \State $set \ p \ visited$
            \For{$ vertex \ t $ in $G$}
                \If{$p$ and $t$ are directly connected}
                    \If{$min\_cost[t]>min\_cost[p]+G[p][t]$}
                        \State $min\_cost[t] \gets min\_cost[p]+G[p][t]$
                    \EndIf
                \EndIf
            \EndFor
        \EndFor
        \State \Return $min\_cost$
        \EndFunction
    \end{algorithmic}
\end{algorithm}
\newpage \quad

\begin{algorithm}
    \caption{Path\_Scanning}
    \begin{algorithmic}[1]
        \Require $Graph\ G,Edge\_cost\ E$
        \Ensure  $solution$
        \Function{path\_scanning}{$G,E$}
            \State $solution \gets [\ ]$
            \For{$i$ in range vehicles's number}
                \State $arcs[] \gets free\ arcs$
                \State $route \gets [\ ],pos \gets 1,cap \gets capacity$
                \While{$True$}
                    \State$ d\gets 9999999$
                    \For{$ arc $ in arcs \& $cap-G[arc]>0$ }
                        \If{$dis[pos][beg(arc)]<d$}
                            \State$ d \gets dis[pos][beg(arc)]$
                            \State$ u \gets arc$
                        \ElsIf{$dis[pos][beg(arc)]=d$}
                            \If{$G[arc]/E[arc]>G[u]/E[u]$}
                                \State $u \gets arc$
                            \EndIf
                        \EndIf
                    \EndFor
                    \If{d doesn't change}
                    \State break
                    \EndIf
                    \State delete u and u\^ in $arcs$
                    \State route.append(u),pos=beg(u),cap-=G[u]
                \EndWhile
                \State solution.append(route)
            \EndFor
            \State \Return $solution$
        \EndFunction
    \end{algorithmic}
\end{algorithm}
\newpage \quad

\begin{algorithm}
    \caption{Sub\_Thread}
    \begin{algorithmic}[1]
        \Require $solution \ S$
        \Ensure $best \ solution \ B$
        \Function{sub\_thread}{S}
            \State $t \gets fitness(S)$
            \State $best \gets S$
            \State $local \gets S$
            \State $temperature \gets t^2$
            \State $break\_time \gets 8 \ times \gets 0$
            \While{$temperature>0.01$}
                \If{$times >= break\_time$}
                    \State break
                \EndIf
                \For{$i$ in range(4)}
                    \State new\_solution $\gets$ creat\_new\_route(local)
                    \State a $\gets$ fitness(new\_solution)
                    \State b $\gets$ fitness(local)
                    \If{$a<b$}
                        \State $local \gets new\_solution$
                    \ElsIf{$exp((a-b)/temerature)> random(0.0,1.0)$}
                        \State $local \gets new\_solution$
                    \EndIf
                \EndFor
                \If{$fitness(local)<fitness(best)$}
                    \State $best \gets local$
                \Else
                    \State $times \gets times+1$
                \EndIf
                \State $temperature*=0.85$
            \EndWhile
            \State \Return $best$
        \EndFunction
    \end{algorithmic}
\end{algorithm}

\begin{algorithm}
    \caption{MS}
    \begin{algorithmic}[1]
        \Require $solution \ S$
        \Ensure $ a \ new \ solution \ N$
        \Function{MS}{S}
            \State $p \gets random \ integer$
            \State sub\_route\_set $\gets$ select p route in S
            \State apply path\_scanning on sub\_route\_set
            \State modify the solution S get N
            \If{$N \ is \ valid $}
                \State \Return $N$
            \Else
                \State \Return $S$
            \EndIf
        \EndFunction
    \end{algorithmic}
\end{algorithm}
\newpage \quad


\section{Empirical Verification}
For this project, it is easy to do the empirical verification. Since we already have the test sets, we can use them and calculate the cost of the final solution the algorithm return and compare the final solution's quality with the original solution's quality to judge how much the solution has improved.

\subsection{Design}
To design test is pretty simple due to the only thing we need to do is the type right input in the Terminal and wait it give a final solution and its total cost
\subsection{Data}
The Data set I use in test is what teacher has offered on Sakai. Because the data set given by teacher is enough for testing , I did not find other data sets on Internet.
\subsection{Performance}
Teacher offers 7 data sets on this project and I will show the result on all of them.The test environment uses i5-7300HQ 2.50GHZ, 8GB RAM, and use all the 4 processing to run the code.

\quad \\


\begin{tabular}{l}
\hline
gdb1.dat  RUNNING TIME: 60s RANDOM SEED: 2\\
\hline
s 0,(12,7),(7,8),(8,10),(10,11),(11,5),0,0,(1,12),(12,6),\\
(6,7),(7,1),0,0,(1,4),(4,3),(3,2),(9,11),(11,8),0,0,(12,5),\\
(5,3),(5,6),0,0,(1,2),(2,4),(2,9),(9,10),(10,1),0\\
q 316\\
%the original solution's cost is 350\\
\hline
\end{tabular}

\quad \\

\begin{tabular}{l}
\hline
gdb10.dat  RUNNING TIME: 60s RANDOM SEED: 2\\
\hline
s 0,(1,10),(10,12),(10,9),(9,3),(3,4),(4,2),(2,1),0,0,(1,4),\\
(4,6),(6,11),(11,12),(12,8),(8,1),0,0,(1,11),(11,4),(4,7),\\
(7,2),(2,5),(5,1),0,0,(1,9),(9,8),(8,3),(3,2),(7,5),(5,6),0\\
q 275\\
%the original solution's cost is 304\\
\hline
\end{tabular}

\quad \\


\begin{tabular}{l}
\hline
val1A.dat  RUNNING TIME: 60s RANDOM SEED: 2\\
\hline
s 0,(1,11),(11,6),(6,7),(7,8),(8,14),(14,13),(13,17),\\
(17,18),(18,14),(13,7),(17,12),(12,6),(6,5),(5,11),\\
(11,12),(12,16),(16,17),(16,15),(20,21),(21,24),(24,23),\\
(23,22),0,0,(1,20),(20,15),(15,21),(21,23),(22,19),\\
(19,20),(1,19),(19,9),(9,3),(3,10),(10,2),(2,4),\\
(4,3),(9,1),(1,5),(5,4),(2,5),0\\
q 183\\
%the original solution's cost is 188\\
\hline
\end{tabular}

\quad \\

\begin{tabular}{l}
\hline
val4A.dat  RUNNING TIME: 60s RANDOM SEED: 2\\
\hline
s 0,(2,8),(8,7),(8,9),(9,14),(14,13),(23,24),(24,25),\\
(25,31),(32,26),(26,27),(27,28),(28,22),(22,21),(21,27),\\
(27,33),(33,37),(27,32),(27,20),(20,21),(22,18),(18,17),\\
(17,20),(20,19),(19,16),(16,15),0,0,(1,2),(2,3),(3,4),\\
(4,5),(5,6),(6,12),(12,11),(11,17),(17,16),(16,11),(11,5),\\
(4,10),(10,11),(10,15),(15,25),(25,26),(26,19),(15,14),\\
(14,24),(24,30),(23,14),(10,9),0,0,(1,7),(7,13),(13,23),\\
(23,29),(29,34),(34,38),(38,39),(39,40),(40,36),(36,39),\\
(39,35),(35,36),(36,37),(37,41),(41,40),(36,32),(32,31),\\
(31,35),(35,34),(31,30),(30,29),(9,3),0\\
q 428\\
%the original solution's cost is 431\\
\hline
\end{tabular}



\begin{tabular}{l}
\hline
val7A.dat  RUNNING TIME: 60s RANDOM SEED: 2\\
\hline
s 0,(1,11),(11,16),(16,15),(15,9),(9,10),(10,16),(15,14),\\
(14,8),(8,9),(10,1),(1,35),(35,36),(36,2),(2,6),(6,1),(1,40),\\
(40,8),(8,1),0,0,(1,33),(33,26),(27,34),(34,35),(35,28),\\
(28,27),(27,26),(26,34),(34,33),(11,12),(17,20),(20,21),\\
(21,22),(22,25),(25,24),(24,23),(23,20),(21,24),(22,18),\\
(18,17),(4,38),(38,39),0,0,(1,2),(2,3),(3,4),(4,5),(5,39),\\
(39,32),(32,31),(31,30),(30,29),(29,36),(36,37),(37,38),\\
(38,31),(31,37),(37,30),(37,3),(3,7),(7,13),(13,17),(17,12),\\
(12,6),(6,7),(13,18),(18,19),(19,13),(13,12),0\\
q 304\\
%the original solution's cost is 326\\
\hline
\end{tabular}


\begin{tabular}{l}
\hline
egl-e1-A.dat  RUNNING TIME: 60s RANDOM SEED: 2\\
\hline
s 0,(60,61),(60,62),(62,66),(66,68),(62,63),(63,65),(56,55),\\
(41,35),(35,32),0,0,(44,46),(46,47),(47,48),(47,49),(51,21),\\
(21,22),(22,75),(75,23),(23,31),(31,32),(32,34),(32,33),0,0,\\
(1,2),(2,3),(2,4),(4,5),(9,10),(11,59),(59,69),(69,4),0,0,\\
(11,12),(12,16),(16,13),(13,14),(15,17),(15,18),(18,19),\\
(19,21),(51,49),(49,50),(50,52),(52,54),(19,20),(20,76),0,0,\\
(69,58),(58,59),(59,44),(44,45),(44,43),(42,57),(57,58),\\
(58,60),0\\
q 3839\\
%the original solution's cost is 4263\\
\hline
\end{tabular}



\begin{tabular}{l}
\hline
egl-s1-A.dat  RUNNING TIME: 60s RANDOM SEED: 2\\
\hline
s 0,(66,67),(67,68),(67,69),(69,71),(71,72),(72,73),(73,44),\\
(44,45),(45,34),(44,43),(124,126),(126,130),0,0,(104,102),\\
(66,62),(62,63),(63,64),(55,140),(140,49),(49,48),(139,33),\\
(33,11),(139,34),0,0,(1,116),(116,117),(117,2),(117,119),\\
(118,114),(114,113),(113,112),(112,107),(107,110),(110,111),\\
(110,112),0,0,(107,108),(108,109),(107,106),(106,105),0,0,\\
(64,65),(55,54),(11,8),(8,6),(6,5),(8,9),(13,14),(13,12),\\
(12,11),0,0,(105,104),(95,96),(96,97),(97,98),(56,55),(11,27),\\
(27,25),(25,24),(24,20),(20,22),(27,28),(28,30),(30,32),\\
(28,29),0,0,(87,86),(86,85),(85,84),(84,82),(82,80),(80,79),\\
(79,78),(78,77),(77,46),(46,43),(43,37),(37,36),(36,38),\\
(38,39),(39,40),0\\
q 5428\\
%the original solution's cost is 6382\\
\hline
\end{tabular}

\quad \newpage
\subsection{Improvement}
Now let's compare the performance improvement between original solution and the final solution.

\quad \newline

\begin{tabular}{|c|c|c|c|c|}
  \hline
    & gdb1 & gdb10 & val1A & val4A  \\
  \hline
  original & 350 & 304 & 188 & 431 \\
  \hline
  final & 316 & 275 & 183 & 428  \\
  \hline
  improvement & 9.7\% & 9.5\% & 2.6\% & 0.69\% \\
  \hline
\end{tabular}

\quad \newline

\begin{tabular}{|c|c|c|c|}
  \hline
    & val7A & egl-e1-A & egl-s1-A \\
  \hline
  original &  326 & 4263 & 6382 \\
  \hline
  final &  304 & 3839 & 5428 \\
  \hline
  improvement &  6.7\% & 9.9\% & 14.9\% \\
  \hline
\end{tabular}




\subsection{Analysis}
From the result above we can find, for some small scale data set, my algorithm can easily find the optimal solution, but for these medium scale data set, my algorithm seems to be trapped at the local optimal and makes little improvement. But for the two large scale data set, my algorithm can improve a lot compared to the original solution. One reasonable explain is due to the scale of data set. The range of local search for better solution changes according to size, for medium scale data set, the range is small and my algorithm can not jump out the local optimal trap. While for the large size data set, the search space is very large so I can find a better solution.To draw a conclusion, the key problem in my algorithm is that it can not generate a solution to jump out of the local optimal trap.
\ifCLASSOPTIONcompsoc

\bibliographystyle{IEEEtran}
\begin{thebibliography}{1}
\bibitem{reference}
\begin{CJK*}{GBK}{song}
���컪, �. CARP�����Ԫ����ʽ�㷨����[J]. 2013.
\end{CJK*}
\bibitem{reference}
Tang K, Mei Y, Yao X. Memetic Algorithm With Extended Neighborhood Search for Capacitated Arc Routing Problems[J]. IEEE Transactions on Evolutionary Computation, 2009, 13(5):1151-1166.

\end{thebibliography}




\end{document}


