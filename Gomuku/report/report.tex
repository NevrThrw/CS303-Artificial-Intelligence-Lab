\documentclass[conference,compsoc]{IEEEtran}
\usepackage{CJK}
\ifCLASSOPTIONcompsoc
  \usepackage[nocompress]{cite}
\else
  \usepackage{cite}
\fi
\ifCLASSINFOpdf
\else
\fi
\usepackage{amsmath}
\usepackage{algorithm}
\usepackage{algorithmicx}
\usepackage{algpseudocode}
\renewcommand{\algorithmicrequire}{\textbf{Input:}}
\renewcommand{\algorithmicensure}{\textbf{Output:}}
\usepackage{url}
\hyphenation{op-tical net-works semi-conduc-tor}
\begin{document}
\title{The Gomoku AI}
\author{\IEEEauthorblockN{Peijun Ruan  11610214}
\IEEEauthorblockA{School of Computer Science and Engineering\\
Southern University of Science and Technology\\
Email: 11610214@mail.sustc.edu.cn}}
\maketitle
\IEEEpeerreviewmaketitle
\section{Preliminaries}
This is a project of Artificial Intelligence,which aims at implementing a simple ai of Gomoku. Gomoku also called five in a row, is a board game for two players, in which the aim is to place five pieces with same color in a row without block.As a game with complete information, people obviously want to calculate all the step to guarantee to win. Considering the huge calculation that need to deal with, even predicting 5 steps is  very hard. In this project, I choose to implement a one-step algorithm.
\subsection{Software}
This project is implemented in Python using IDE PyCharm. The libraries being used includes Numpy,Re and Random.
\subsection{Algorithm}
The algorithm used in this project is mainly heuristic search.The main method to implements this search algorithm and some other functions is iteration.There totally have five functions in this project's Python code file.
\section{Methodology}
The rule of Gomoku is very simple,you just need to try your best to place five stone in the same row and avoid your opponent win ahead of you.This part will show how I implement the main structure of the Gomoku AI.
\subsection{Representation}
First of all,it is unavoidable to consider how to represent the data you use in playing gomoku, such as black stone,white stone the chessboard, the choice your program makes at last and so on.The following list show some data structure I use to represent data and the chess patterns I use.
\begin{itemize}
\item \textbf{Chessboard}: a numpy two-dimensional array.
\item \textbf{List of candidate position}: list of tuple.
\item \textbf{Output position}: the last tuple in the candidate list.
\item \textbf{Chess color}: the color of stone:
    \begin{itemize}
        \item \large color\_white: use 1 to represent
        \item color\_black: use -1 to represent 
    \end{itemize}
\item \textbf{Chess patterns}[1]:
    \begin{itemize}
        \item \large five:300000 pts
        \item live\_four:100000 pts
        \item cut\_four:30000 pts
        \item live\_three:15000 pts
        \item cut\_three:5000 pts
        \item live\_two:3000 pts
        \item cut\_two:1000 pts
    \end{itemize}
\end{itemize}
\subsection{Architecture}
Here are all functions in Python file gobang with pesudocode format.
\begin{itemize}
\item \textbf{Given}:
    \begin{itemize}
        \item \_\_initial\_\_:initial the AI.
        \item go:read a chessboard and output the decision.
    \end{itemize}
\item \textbf{Self define}:
    \begin{itemize}
        \item get\_candidate\_simple: find all the valid position.
        \item check\_neighbors: check whether in a 4*4 field exists a stone.
        \item get\_score: count the score if AI places stone here.
        \item get\_stone\_around: count stones of mine and opponent's.
    \end{itemize}
\end{itemize}

\subsection{Detail of Algorithm}
Here show the psudocode of all the functions. \\
%\vspace{-20pt}
\begin{itemize}
\item \textbf{check\_neighbors:}Use transversal to search the chessboard, if there is no stone around this 
position, it is useless to place a stone here
\end{itemize}
%\vspace{-30pt}
\begin{algorithm}
    \caption{check whether in a 4*4 field exists a stone.}
    \begin{algorithmic}[1]
        \Require $tuple$ index,$np.array$ chessboard
        \Ensure $true \ or false$
        \Function{check\_neighbors}{$chessboard,index$} \\
         $ x,y \gets index[0],index[1]$
        \For{$i=-3 \to 3$}
            \For{$j=-3 \to 3$}
                \If{$i==0 \& j==0$}
                \State continue
                \ElsIf{$0<=x+i<=14\ \& \ 0<=y+j<=14\ \& \ (x+i,y+j)\ is\ empty $}
                \State \Return{$true$}
                \EndIf
            \EndFor
        \EndFor
        \State \Return{$false$}
        \EndFunction
    \end{algorithmic}
\end{algorithm}

\begin{itemize}
\item \textbf{get\_candidate\_simple:}Search the whole chessboard to find all valid position
\end{itemize}

\begin{algorithm}
    \caption{find all the valid position.}
    \begin{algorithmic}[1]
        \Require $np.array$ chessboard
        \Ensure $list$ candidatelist
        \Function{get\_candidate\_simple}{$chessboard$} \\
        $list \gets [ \ ]$
        \For{$i=0 \to 14$}
            \For{$j=0 \to 14$}
                \If{$chessboard[i][j] \ is \  empty \, \newline
                \& \, check\_neighbors(chessboard,(i,j))$}
                \State $list.add [i,j]$
                \EndIf
            \EndFor
        \EndFor
        \State \Return{$candidatelist$}
        \EndFunction
    \end{algorithmic}
\end{algorithm}
%\vspace{-20pt}
\begin{itemize}
\item \textbf{get\_score:} Check all the chess pattern if place a stone here to measure the importance of this position
\end{itemize}
\quad \newline
%\vspace{10pt}
\begin{algorithm}
    \caption{calculate the score of chess pattern.}
    \begin{algorithmic}[1]
        \Require $np.array$ chessboard,$tuple$ index,$int$ color
        \Ensure $int$ score
        \Function{get\_score}{$chessboard,index,color$} \\
        $x,y \gets index[0],index[1]$ \\
        $score \gets 0$ \\
        $pattern\_list \gets \{ \ \}$ \\
        $position \gets \{left\_to\_right,up\_to\_down, \newline
        leftdown\_to\_rightup,leftup\_to\_rightdown\}$
        \For{$i$ \ in \  $position$}
            \State find chess pattern $p$ by searching $i$ at $(x,y)$
            \If{$p \ in \ Chess \ patterns$}
                \State $score += p's \ score$ 
                \State $pattern\_list[p]+=1$ \\ // record the number and type of pattern
            \EndIf
        \EndFor
        \State \Return $score,pattern\_list$
        \EndFunction
    \end{algorithmic}
\end{algorithm}

%\vspace{-10pt}
\begin{itemize}
\item \textbf{get\_stone\_around:} Search how many stone around here. more concentrated the stones are located , more likely to get advantage
\end{itemize}
%\vspace{10pt} 
\begin{algorithm}
    \caption{count stones of mine and opponent's.}
    \begin{algorithmic}[1]
        \Require $np.array$ chessboard,$tuple$ index, $int$ color 
        \Ensure $enemy$,$mine$ 
        \Function{get\_stone\_around}{$chessboard,index, \newline
        color$} \\
        $x,y \gets index[0],index[1]$ \\
        $enemy,mine \gets 0$
        \For{$i=-1 \to 1$}
            \For{$j=-1 \to 1$}
                \If{$i==0 \ \& \ j==0$}
                \State continue
                \ElsIf{$14>=x+i>=0 \ \& \ 14>=y+j>=0$}
                    \If{$chessboard[x+i][y+j]==-color$}
                    \State enemy+=1
                    \ElsIf
                    \State \quad mine+=1
                    \EndIf
                \EndIf
            \EndFor
        \EndFor
        \State \Return $enemy,mine$
        \EndFunction
    \end{algorithmic}
\end{algorithm}

\newpage
\begin{itemize}
\item \textbf{go:} Given a chessboard, search the whole board to determine which position will get the largest benefit.
\end{itemize}
\begin{algorithm}
    \caption{output the final decision position.}
    \begin{algorithmic}[1]
        \Require $np.array$ chessboard
        \Ensure the last tuple in candidatelist
        \Function{go}{$chessboard$}  \\
        $list \gets get\_candidate\_simple(chessboard)$ \\
        $maxscore \gets -1$
        $candidatelist \gets [ \ ]$
        \For{$ i \ in list$} 
            \State $attack,attack\_pattern \gets get\_score(chessboard,i,color)$
            \State $defence,defence\_pattern \gets get\_score(chessboard,i,-color)$
            \If{$attack\_pattern \ and \ defence\_pattern \newline \ can \ consist \ of \ some \ special \ pattern$}
                \State add extra score for those special chess pattern to attack and defence
            \EndIf
            \State $value \gets attack+defence$
            \If{$value>maxscore$}
                \State $maxscore \gets value$
                \State $candidatelist.append(i)$
            \EndIf 
        \EndFor
        \EndFunction
    \end{algorithmic}
\end{algorithm}

\section{Empirical Verification}
For empirical verification, using code\_check\_test.py file to check whether the algorithm works and can give the right answer. The result shows my algorithm works well.

\subsection{Design}
The main function in this project is go() and I implement other four functions to assist go() to meet the goal that give a decision where to place the next stone.And among the four functions(except go function), get\_stone\_around is an extra function aims at reducing useless computing and decrease the running time.
\subsection{Data and data structure}
Data used in this project is some chess\_log files to improve the details and fix bugs.
Data structures used include tuple,list,dictionary,array. 
\subsection{Performance}
Instead of using game tree, this AI is simply designed using whole-chessboard-transversal.The time complexity will not be very high,which is about $\mathcal{O}(n^2)$.This AI can make decision very quickly compared to game tree algorithm.
\subsection{Result}
See the score and rank on website \textbf{10.20.96.148}. The score suggests my algorithm works not bad.
\subsection{Analysis}
This AI project is just a simple realization and not even use some advanced algorithms like game tree, Monte Carlo tree. Compared with those AI, this only consider a few situation. Thus, it may perform not so well as those using game tree and it can not handle some special opening like D4,I7. But it is also unreal to implement an AI that can deal with all possible situation by using the tools and data in our hands. In a word, this kind of simple AI's performance absolutely depends on how well you know about Gomoku, and it reminds me that it is a long way to program a true AI.
\ifCLASSOPTIONcompsoc

  \section*{Acknowledgments}
I would like to thank my classmates Kai Lin and Ziyuan Ye who give me inspiration on the AI's logic and inform me some knowledge on Gomoku game, which helps me improve the AI's performance a lot.Last I would like to thank SA who will check my codes and reports.
\else
  \section*{Acknowledgment}
\fi

\bibliographystyle{IEEEtran}
\begin{thebibliography}{1}
\bibitem{reference} 
\begin{CJK*}{GBK}{song}
Cnblogs.com.(2018).������AI��˼·-��������-����԰.[online]Available at: \\ https://www.cnblogs.com/songdechiu/p/5768999.html [Accessed 27 Oct. 2018].
\end{CJK*}
\end{thebibliography}


\end{document}


